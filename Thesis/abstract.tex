Motion Sickness is a common problem in Virtual Reality (VR).
It is often described as discontinuity between seen motion and motion felt by the vestibular system, though multiple other factors play a role.
While much research has been done to understand why motion sickness occurs and when it is happening in the user, little has been done to prevent motion sickness as or before it occurs.
We plan to use a machine learning algorithm to detect when a user is likely experiencing motion sickness.
Once motion sickness has been detected, corresponding mitigation features such as vignetting, snap turning, or user warnings are enabled on the user's behalf.
We believe this will reduce instances of reported motion sickness making the user experience more enjoyable, while also allowing the user to remain in vr for longer. % You didn't test this, change.





