% ======================================================================
% Example Wentworth MSACS LaTeX Thesis.
% ======================================================================
%
% (Lines beginning with % are comments and are ignored.)
% 
% The class file wit-thesis.cls must be in the current directory or
% installed with the other classes as per standard LaTeX installation.
% 
% To generate run these commands:
%    latex  thesis
%    bibtex thesis
%    latex  thesis
%    latex  thesis
% Then you need to use the dvips command to get postscript output
% 
% See the README file for more information
% 


\documentclass{wit-thesis}
% 
% For early printouts to save paper use the savepaper option as
% 
% \documentclass[savepaper]{wit-thesis}
% 
% This will make things single spaced, use small font and smaller
% margins.  Stuff will be formatted differently if you don't use this
% option but it's useful to basically see (read) what you typed so far
% on paper without wasting much paper.  You might want to also comment
% out the front matter and backmatter if printing out in savepaper
% mode to save paper there.  Do not use this option on your final
% printout as it doesn't satisfy the thesis manual requirements.

% Also if you want to use double spacing rather then singlespacing (if
% your thesis is very short, say 25 pages or less), then use the
% `doublespace' option as
% 
% \documentclass[doublespace]{thesis-thesis}


% ======================================================================
% Graphics and Figures
% ======================================================================
%
% For including graphics use
% (Info) http://en.wikibooks.org/wiki/LaTeX/Importing_Graphics
%
% NOTE: My *may* need the graphicx package to get the correct
% page-size (letter) for your document... Some environments,
% e.g. TeXnicCenter default to 'a4' page size.
\usepackage{epsfig}
\usepackage{graphicx}

% These packages may also be useful for pictures...
% \usepackage{color}
% \usepackage{eepic}
% \usepackage{epic}
% \usepackage{grapic}
%
% NOTE: adding packages is fun and exciting, but OFTEN has unintended
%       consequences, as they redefine "things."  To minimize your
%       pain, minimize the number of extra packages you include.
%       Remember that you are trying to GRADUATE, not trying to
%       generate The Ultimate Thesis to Blow Them[tm] All Away.


% ======================================================================
% The American Mathematical Society (AMS) packages
% ======================================================================
% You quite likely want these if you have equations in your thesis:
\usepackage{amsmath}
\usepackage{amsfonts}
\usepackage{amssymb}
\usepackage{amsthm}


% ======================================================================
% LONGTABLE is rarely used, but hard to get right... so I include it
% in the right place, just to be safe.
% ======================================================================

\usepackage{longtable}


% ======================================================================
% This makes captions *bold*
% ======================================================================
%
%(NOTE): Adding "justification=justified,singlelinecheck=false" to the
%        list of options will force all captions (including "short
%        one-line") to be left justified.  This is technically
%        required by the DTM, but looks ugly.
\usepackage[bf,labelsep=period,textfont=bf]{caption}


% Other useful packages for theses (see LaTeX docs for descriptions of these)
% 
% For the \vref commands that also prints out the reference page
% \usepackage{varioref}
% 
% For including computer code
% \usepackage{alltt}
% 
% For the \url{http://foo.com} command to include url's (or filenames)
% \usepackage{url}
% 

% ======================================================================
% This package countains the \sout command (which you should never
% use!)
% ======================================================================
\usepackage[normalem]{ulem}

% To add an in-text degree symbol.
\usepackage{textcomp}

% For refrence links and hyperlinks
\usepackage{hyperref}

% For SVG file compatibility
\usepackage{svg}

% for tables
\usepackage{booktabs}
\usepackage{tabularray}
\usepackage{tabularx}

% ======================================================================
% ======================================================================
% You definitely need to edit things BELOW this line
% ======================================================================
% ======================================================================

% Author name
\author{Jacob L. Ledbetter}


% Title of the thesis (all in upper case), use \\ for line breaks as
% usual, you can use up to 4 lines and make sure to set the counter
% titlelines to the number of lines you used.
% 
% This is for the title page
% 
\title{A Method For Active Motion Sickness Reduction Using Predictive Models}
% Number of lines in the title, without setting this the title page
% will not be formatted properly
\setcounter{titlelines}{1}


% Heading style title, the number of lines can be different here then
% in titlelines. This is for the abstract pages and the signature page.
% 
% (FORMAT) Make sure that this title has the EXACT same words at the
% (FORMAT) title-page-title
% 
\titleheading{A Method For Active Motion Sickness Reduction Using Predictive Models}


% (FORMAT) The "degree" is set on three lines; select one of the
% following formats.

% Degree (MS-ACS)
\degreeONE{Master of Science}
\degreeTWO{in}
\degreeTHREE{Applied Computer Science}



% ======================================================================
% If you need to change the word 'Thesis' use \thesisname{Blah} and if
% you need to change the middle line between \degree and \degreein on
% the titlepage to something other then 'in' use \inofand{of} to use
% 'of' for instance.  (This should not be necessary)
% ======================================================================


% Dates
\gradyear{2024}
% (Format) Term Year 
\submitdate{Spring 2024}


% ======================================================================
% Thesis Committee
% ======================================================================
%
% Your committee chair (don't include titles)
% (FORMAT) Do not include the institution ("WIT") for local faculty
% (FORMAT) members; however, for external members DO include the
% (FORMAT) institution.
\committeechair{Yetunde Folajimi}
\committeechairdept{School of Computing and Data Science}

% Second committee member
\committeesecond{Micah Schuster}
\committeeseconddept{School of Computing and Data Science}

% Third (usually different department) committee member
\committeethird{Lauren Melfi}
\committeethirddept{School of Computing and Data Science}


% ======================================================================
% This is the start of the document
% ======================================================================
\begin{document}

% Title page 
% (FORMAT) Mandatory for thesis
\maketitle

% Signature page
% (FORMAT) Mandatory for thesis
\makesignature

% Copyright page
% (FORMAT) Mandatory for thesis
\begin{copyrightpage}
  Copyright~\copyright~2024\\
  by\\
  Jacob L. Ledbetter
\end{copyrightpage}


% ======================================================================
% Dedication (make sure to format this correctly including a vspace
% (say \vspace{3in} or using vfill) to make it center on the page if
% desired, see the thesis manual) Or just delete this if you don't
% have a dedication
% 
% (FORMAT) Optional page
\begin{dedication}
  \vspace{3in}
  \centering
%  Dedicated to the voices in my head, for being more sane than anyone else I know(including me)
  Dedicated to my grandmother, may she rest in peace.
\end{dedication}


% ======================================================================
% Epigraph (make sure to format this correctly, it will just be
% centered on the page, see the manual) Or just delete this if you
% don't have an epigraph
% 
% (FORMAT) Optional page
%\begin{epigraph}
%  We must know, we shall know.\\
%  \begin{flushright}
%    -- David Hilbert
%  \end{flushright}
%\end{epigraph}


% ======================================================================
% Here type the abstract of your thesis.
% (FORMAT) Mandatory for thesis
\begin{abstract}
  % This just inserts the the abstract.tex file
  Motion Sickness is a common problem in Virtual Reality (VR).
It is often described as discontinuity between seen motion and motion felt by the vestibular system, though multiple other factors play a role.
While much research has been done to understand why motion sickness occurs and when it is happening in the user, little has been done to prevent motion sickness as or before it occurs.
We plan to use a machine learning algorithm to detect when a user is likely experiencing motion sickness.
Once motion sickness has been detected, corresponding mitigation features such as vignetting, snap turning, or user warnings are enabled on the user's behalf.
We believe this will reduce instances of reported motion sickness making the user experience more enjoyable, while also allowing the user to remain in vr for longer. % You didn't test this, change.






\end{abstract}


% ======================================================================
% Table of contents
% (FORMAT) Mandatory for thesis
\tableofcontents


% ======================================================================
% If you don't want a list of tables page, delete or comment out this
% line
% (FORMAT) ONLY delete this page if you have *no* tables
\listoftables


% ======================================================================
% If you don't want a list of figures page, delete or comment out this
% line
% (FORMAT) ONLY delete this page if you have *no* figures
\listoffigures


% Your acknowledgments go here
% Or just delete this if you don't have acknowledgments
% (you should! - Suck up to your advisor and committee!!!)
\begin{acknowledgments}
  I would like to thank Dr.~Folajimi for helping find articles I could not during my lit review.\\\\
  I would like to thank Dr.~Schuster for guidance, assistance with~\LaTeX, and being a shoulder to cry on when everything goes wrong.\\\\
  I would like to thank Dr.~Melfi for guidance with some complicated matrix transformations.\\\\
  Lastly, I would like to thank members of the ChilloutVR Modding Group Discord server for helping with C\# issues and pardoning some truly terrible code.

\end{acknowledgments}



% 
% This includes body.tex
% 
% Note that if you want something in single space you can go back and
% forth between single space and normal space by the use of \ssp and
% \nsp.  If you want doublespacing you can use \dsp.  \nsp is normally
% 1.5 spacing unless you use the doublespace option (or savepaper
% option)
%
%(FORMAT) Usually you *don't* want to mess with the spacing for your
%(FORMAT) final version.  If you think/know that the thesis template
%(FORMAT) and/or thesis style file is incorrect/incomplete, PLEASE
%(FORMAT) contact the maintainer.  THANK YOU!!!

\chapter{INTRODUCTION}
\label{ch:intro}
% By labeling the chapter, I can refer to it later using the
% label. (\ref{chap:intro}, \pageref{chap:intro}) Latex will take care
% of the numbering.


Virtual Reality (VR) is a popular medium for user interaction.
Its primary feature is the use of a head mounted display(HMD) that allows the user to see a fully rendered virtual environment.
Such a display necessarily blocks the user's vision of the real world.
In addition to the HMD, the user typically has controllers allowing them to move through and interact with the environment.
VR is used in several contexts from telepresence, to simulation and training, to entertainment. % Might need a citation here.

\section{Problem Statement}
\label{sec:problem_statement}

A common problem in VR is that of motion sickness.
Motion Sickness includes symptoms like nausea, fatigue, and upset stomach, among others.
This most often occurs in VR when the user is moved in the virtual environment while the user is stationary in reality.
Examples include moving using the controllers, standing on moving platforms, or riding vehicles.

Many VR applications are aware of these issues and have implemented some features to prevent motion sickness.
One common option is to apply a vignette to the user's vision when they are moving.
Another is to change how the user moves, either by letting the user select an area or move an avatar in third person to an area, and then teleporting them to said specified area.
The last common option is to snap the user's rotation a set increment rather than smoothly turn them.

No solutions currently exist to dynamically reduce motion sickness for VR users based on the user's level of motion sickness.

\section{Literature Review}
\label{sec:lit_review}

The majority of solutions to reduce motion sickness are either user toggleable features, like those above, or attempts to reduce latency within the VR system\cite{kundu2021study} (where latency can cause unexpected environment changes and thus sickness).
User toggleable features either require the user to toggle them (before or after becoming motion sick), or are on by default, possibly degrading the user's experience where unnecessary.
Decreased latency helps in many cases but is rarely the only cause of sickness.
Many studies have been done to ascertain why people become motion sick in VR\cite{9236907,9133071}.
Some have found that 360\textdegree treadmills can provide relief\cite{10.1117/12.2626662}, but they are not economical for most users, and do not solve all potential motion sickness triggers.
Some have also found that visual cues can reduce motion sickness\cite{10.1145/3544999.3552489}, but if unsubtle, may degrade the user experience, especially for those who do not need such cues.

A variety of models have been made to predict motion sickness in the user.
Some use Deep Neural Networks (DNNs)\cite{8613651}, some use Convolutional Neural Networks (CNNs)\cite{8642906}, while others use more classical Machine Learning (ML) approaches\cite{8267239}.
Most models predict using a combination of data from the application as well as user submitted feedback, but others predict using the users themselves, specifically their eye movements\cite{9234030,https://doi.org/10.1111/cgf.14703}.
Some tools have even been made to help in the creation of predictive models\cite{10.1145/3526113.3545656}.
However, none of these models have been used to drive the reduction of motion sickness, they solely predict when/if the user is getting motion sick.

\setion{Objectives}
\label{sec:objectives}

The goal is to create a model that can predict a user becoming motion sick and find the most applicable motion sickness reduction feature.
It should then activate the motion sickness reduction, and subsequently deactivate it, once the user is no longer at risk of motion sickness.
As stated in section~\ref{sec:lit_review}, there are many models present for predicting motion sickness, thus many options to consider for what kind of model to use.

\chapter{METHODS}
\label{ch:methods}

This is the methods, IE how I did it(badly).

\section{Motion Sickness Detection with Convolutional Neural Networks}
\label{sec:detection}

I believe a CNN would be most useful as it can take data from the viewpoint of the user as well as other data (I.E.\ controller, motion, and pose data) from the user.
A CNN take in a wide array of information at once, a CNN can categorize into multiple motion sickness categories, causes, or ratings beyond a binary ``is motion sick'' or ``is not motion sick'' rating.
More importantly, CNNs can convolve over a window of time to provide a more accurate assessment of the user's sickness as it changes over time.

\subsection{Dataset}
\label{subsec:dataset}

This is my dataset.
Youch.

\subsection{Network Architecture}
\label{subsec:architecture}

Here's how the Net is set up.

\subsection{Training}
\label{subsec:training}

Here's how I trained the model.

\section{Motion Sickenss Prevention with Vignetting, Snap Turning, and User Warnings}
\label{sec:prevention}

How am I preventing motion sickness?

\subsection{Vignetting}
\label{subsec:vignetting}

Blind 'em.

\subsection{Snap Turning}
\label{subsec:snap_turn}

You will not rotate.

\subsection{User Warnings}
\label{subsec:user_warnings}

You will be warned.

\section{Marrying Detection and Prevention}
\label{sec:marrying}

Putting them together was more complicated than you might think.

\subsection{The Environment}
\label{subsec:environment}

I made a roller-coaster.
It's pretty cool.
Sick even.

\subsection{Loading the Model in Unity}
\label{subsec:loading}

This went horribly.
Curse you Microsoft.

\subsection{Using Inter-Process Communication Instead}
\label{subsec:ipc}

This is bad, terrible.


\chapter{RESULTS}
\label{ch:results}

So here's how it turned out!

\chapter{CONCLUSION}
\label{ch:conclusion}

And finally\ldots

%Intro
%  Problem Statement
%  Objectives
%  Literature Review
%Methods
%  Motion Sickenss Detection with Convolutional Neural Networks
%    Dataset
%	Network Achitecture
%	Training
%  Motion Sickenss Prevention with Vignetting, Snap Turning, and User Warnings
%    Vignetting
%	Snap Turning
%	User Warnings
%  Marrying Detection and Prevention
%    Loading the Model in Unity
%	Using Inter-Process Communication Instead
%Results
%  Model Predictions
%  Model Accuracy
%  Application Performance
%Conclusions
%  Future Work
%Appendex
%Bibliograpy




% 
% The bibliography page must be between main body and appendices
%
% You must have thbib.bib file in the current directory 
%
% (uncomment \nocite{*} to force inclusion of all uncited entries)
% \nocite{*}
\bibliographystyle{IEEEtran}
\bibliography{bibi}

% This includes append.tex
%\appendices
%
% If you only have one appendix, you should change the above to:
%\appendix
%

\chapter{MORE INFORMATION ON EQUATIONS}

To demonstrate how an appendix should be inserted into the thesis we
have provided two appendices. This first appendix illustrates some
more advanced techniques to improve the appearance of your equations.
Below is a system of partial differential equations from a model for
cellular control by an external nutrient. The equations are
complicated and \LaTeX\ tends to allow them to run into each other. To
prevent this we have used the \verb+\vrule+ command to separate
them. Note this is an ordinary \TeX\ command and is not in L.\
Lamport's book \cite{LAM}. Furthermore, we have some complicated
boundary conditions that we needed to align, so we used the array
command, but to get the equations looking right we also needed the
\verb+\dfrac+ command instead of the \verb+\frac+ command. The
equations for our model are as follows:
\begin{eqnarray}
  \dot{U}_1(t) & = & \tilde f(W_1(t-T)) - U_1(t) + \gamma_1U_2(R\sigma,
   t){\vrule width 0in depth .1in},	\nonumber \\
  \dot{W}_1(t) & = & -\hat b_3W_1(t) + \gamma_3W_2(R\sigma,
   t){\vrule width 0in depth .1in},\nonumber \\
  \frac{\partial U_2}{\partial t} & = & D_1\nabla^2U_2 - U_2 - \tilde f(W_1
    (t-T)) - \gamma_1U_2(R\sigma,t){\vrule width 0in depth .1in},
	\label{sys2} \\
  \frac{\partial V_2}{\partial t} & = & D_2\nabla^2V_2 - b_2V_2 + c_0
    \bigl(U_2 + U_1(t)\bigr){\vrule width 0in depth .1in}, \nonumber \\
  \frac{\partial W_2}{\partial t} & = & D_3\nabla^2W_2 - b_3W_2 + (\hat b_3
    -b_3)W_1 - \gamma_3W_2(R\sigma,t) \nonumber \\
    &  & + k\left[\left[{\left(\frac{D_3}{r^2}\right)}\frac{d}{dr}\left(r^2
	   \frac{dh}{dr}\right) - b_3h\right]V_2(R,t) - h\dot V_2(R,t)
	   \right], \nonumber
\end{eqnarray}
for $t > 0$ and $R\sigma < r < R$ and with the boundary conditions:
\begin{equation*}
\begin{array}{rclcrcl}
 \dfrac{\partial U_2(R\sigma,t)}{\partial r} & = &
   \beta_1U_2(R\sigma,t), & \qquad &
 \dfrac{\partial U_2(R,t)}{\partial r} & = &
   0, \\
\\
 \dfrac{\partial V_2(R\sigma,t)}{\partial r} & = &
   0, & \qquad &
 \dfrac{\partial V_2(R,t)}{\partial r} & = &
   0, \\
\\
 \dfrac{\partial W_2(R\sigma,t)}{\partial r} & = &
   \beta_3W_2(R\sigma,t), & \qquad &
 \dfrac{\partial W_2(R,t)}{\partial r} & = &
   0.
\end{array}
\end{equation*}
Notice that the system is numbered only once by (\ref{sys2}) and that
this is centered as best we can on one line. All other lines have the
$\backslash$\textit{nonumber} command.


\chapter{LISTS AND QUOTATIONS}

The thesis will rarely use list environments, but they are valuable
for r{\'e}sum{\'e}s. For more information on creating a r{\'e}sum{\'e}
you may want to see the author of this document (you also need to
learn quite a bit about \verb+\parbox+ commands).  To create a list
you will want to use one of \texttt{itemize, enumerate,} or
\texttt{description}. For example:
\begin{description}
\item[continuous] A function $f$ is {\bf continuous} at $x$ if and only
if for every $\varepsilon >0$ there exists a $\delta(x) >0$ such that
whenever $|y-x|<\delta$, $|f(y)-f(x)| < \varepsilon$.
\item[uniformily continuous] A function $f$ is {\bf uniformly
continuous} if and only if for every $\varepsilon >0$ there exists a
$\delta >0$ such that whenever $|y-x|<\delta$, $|f(y)-f(x)| <
\varepsilon$ independent of $x$ and $y$.
\item[equicontinuous] A family of functions $f_n$ is {\bf
equicontinuous} at a point $x$ if and only if for every $\varepsilon >0$
there exists a $\delta >0$ such that whenever $|y-x|<\delta$,
$|f_n(y)-f_n(x)| < \varepsilon$ for all functions $f_n$.
\end{description}

\LaTeX\ provides an environment for block quotations. To agree with standard 
thesis formatting manual follow the format below for a quotation exceeding four
lines. From Lewis Carrol's {\it Hunting of the Snark} we hear the
Bellman tell his crew:
 \vspace{.12pt}

{
\ssp
\begin{verse}
The Bellman himself they all praised to the skies--\\
Such a carriage, such ease and such grace!\\
Such solemnity, too! One could see he was wise,\\
The moment one looked in his face!\\
 \vspace{.15in}
He had bought a large map representing the sea,\\
Without the least vestige of land:\\
And the crew were much pleased when they found it to be\\
A map they could all understand.\\
 \vspace{.15in}
``What's the good of Mercator's, North Poles and Equators,\\
Tropics, Zones, and Meridian Lines?''\\
So the Bellman would cry: and the crew would reply,\\
``They are merely conventional signs!''\\
 \vspace{.15in}
``Other maps are such shapes, with their islands and capes!\\
But we've got our brave Captain to thank''\\
(So the crew would protest) ``that he's bought us the best--\\
A perfect and absolute blank!''\\
\end{verse}
}




\end{document}
